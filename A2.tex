\chapter{Lennard-Jones-Fl�ssigkeit in 2D}
\section{Das Lennard-Jones-Potential}
\label{chap:LJ}
Nun wird die Simulation auf mehrere Teilchen erweitert. Diese Teilchen befinden sich nicht mehr in einem �u�eren Potential sondern wechselwirken mit sich selbst. Das Potential, das jedes von jedem Teilchen an seiner Position erzeugt wird, wird mit Hilfe des Lennard-Jones-Potentials gen�hert. Da das Lennard-Jones-Potential kugelsymmetrisch ist (keine Richtung wird von den Teilchen bevorzugt) l�sst es sich in Abh�ngigkeit von $r$, dem Abstand zum Teilchen, darstellen. Das Lennard-Jones-Potential ist wie folgt definiert:
\begin{equation}
V(r) = 4\epsilon \left[\left(\frac{\sigma}{r}\right)^{12} - \left(\frac{\sigma}{r}\right)^6 \right]
\end{equation}
Das Minimum des Lennard-Jones-Potentials befindet sich bei $r_c=2^\frac{1}{6}\sigma$. Dies entspricht einer Energie von $E_{min}=V(r_c)=-\epsilon$ Die dazugeh�rige Kraftwirkung erh�lt man durch Differenzieren:
\begin{equation}
F(\vec{r}) = -\nabla V(\vec{r}) = 24 \frac{\epsilon}{\sigma^2} \left[\left(\frac{\sigma}{r}\right)^{8} - 2\left(\frac{\sigma}{r}\right)^{14} \right] \vec{r}
\end{equation}
$\epsilon$ und $\sigma$ sind charakteristische Gr��en f�r das Lennard-Jones-Potential, wobei $\epsilon$ eine Energie darstellt und $\sigma$ eine L�nge. Dies wird in Abbildung \ref{fig:LJE} und \ref{fig:LJS} dargestellt.
\begin{figure}[htbp]
	\centering
		\includegraphics[width=0.70\textwidth]{img/LJE.pdf}
	\caption{Verlauf des Lennard-Jones-Potentials in Abh�ngigkeit der Entfernung. Es zeigt sich, dass $\epsilon$ eine charakteristische Energie darstellt.}
	\label{fig:LJE}
\end{figure}
\begin{figure}[htbp]
	\centering
		\includegraphics[width=0.70\textwidth]{img/LJS.pdf}
	\caption{Verlauf des Lennard-Jones-Potentials in Abh�ngigkeit der Entfernung. Es zeigt sich, dass $\sigma$ eine charakteristische L�nge darstellt.}
	\label{fig:LJS}
\end{figure}
Wie in den Diagrammen zu erkennen ist, macht es Sinn, alle Einheiten in der Simulation im Bezug auf $\sigma$ und $\epsilon$ anzugeben. Dies wird damit erreicht, dass in der Simulation $\sigma$ und $\epsilon$ gleich 1 gesetzt werden.

\section{Simulation}
Es wird nun eine 2D-Simulation geschrieben, die den zeitlichen Verlauf der Position und der Geschwindigkeit, von $N$ Teilchen ermittelt. Dazu werden die Teilchen zu Beginn auf einem quadratischen Gitter oder einem Dreiecksgitter angeordnet (vgl. \fref{fig:gitter}).
\begin{figure}[htbp]
	\centering
		\includegraphics[width=0.80\textwidth]{img/gitter.JPG}
	\caption{Quadratische (links) und dreieckige (rechts) Anordnung der Teilchen zu Beginn der Simulation. Ausschnitt �bernommen aus \cite{schroeder-turk2012}.}
	\label{fig:gitter}
\end{figure}
Es zeigt sich, dass sich die quadratische angeordneten Teilchen sofort in Richtung der Dreiecksstruktur orientieren. Dies l�sst darauf schlie�en, dass die Dreiecksstruktur energetisch g�nstiger ist, als die quadratische. F�r gro�e Zeiten expandiert das System, da die Energie, wie in \fref{chap:HarmOsziNumLsg} gesehen, aufgrund von numerischen Fehlern stetig ansteigt.


\section{Komplexit�t}
Um die Komplexit�t der Simulation zu ermitteln, werden mehrere Simulationen mit unterschiedlicher Teilchenzahl durchgef�hrt. Alle anderen Parameter bleiben dabei unver�ndert. Es wird die Zeit f�r jeden Integrationsschritt gemessen und ein Mittelwert aus 100 Schritten gebildet. Anschlie�end werden die Daten doppelt logarithmisch aufgetragen und die Komplexit�t mittels eines Fits bestimmt.
\begin{figure}[h!]
	\centering
		\includegraphics[width=0.70\textwidth]{img/LZ.pdf}
	\caption{Laufzeit in Anh�ngigkeit von der Teilchenzahl mit polynomialen Fit. Es zeigt sich eine $N^2$ Abh�ngigkeit. Die ersten Messwerte werden Aufgrund der Abh�ngigkeit von der zu diesem Zeitpunkt vohandenen Rechnerlast verworfen.}
	\label{fig:LZ}
\end{figure}
Die Bestimmung der Steigung der Geraden ergibt \textbf{2.01}. Dies belegt eindeutig die erwartete Komplexit�t von $O(N^2)$. F�r die Bestimmung der Steigung wurden nur die letzten Werte verwendet, da sie die aussagekr�ftigsten sind. Zur Bestimmung der Rechenzeit wurde die Differenz zwischen dem Zeitpunkt zum Beginn des Schrittes und zu Ende des Schrittes genommen. Da aber in modernen Betriebssystemen die Rechenleistung auf verschiedene Anwendungen verteilt wird, ist die Annahme, dass in dieser Zeit nur die Simulation gelaufen ist, nicht zu halten. Um das Messergebnis nicht zu verf�lschen wurden s�mtliche anderen Arbeiten w�hrend der Messung eingestellt. Aber auch die Hintergrundprozesse ben�tigen Rechenleistung. Bei den kleinen Messwerten ist das Verh�ltnis zwischen Hintergrundrechenzeit und Rechenzeit, die wirklich in der Simulation verbraucht wurde, relativ gro�, so dass diese Werte sehr unzuverl�ssig sind. Bei den h�heren Messwerten wird dieser Effekt immer kleiner, da die Rechenleistung, die die Simulation verbraucht, immer dominanter wird.