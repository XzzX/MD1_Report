%Mustervorlage (c) Sebastian Eibl

\documentclass[%
12pt,														% 10pt, 11pt, 12pt W�hlt die Schriftgr��e im Dokument. Standard ist �10pt�.
a4paper,													% a4paper, a5paper, b5paper, letterpaper, legalpaper Legt das Papierformat fest. Standard ist �letterpaper�.
															% landscape W�hlt Querformat f�r das Papier.
DIV=calc, 										% Errechnet einen guten Satzspiegel
%BCOR=1.5cm, 										% Weil bei mir immer 1cm in der Bindung
titlepage,													% titlepage, notitlepage Legt fest, ob es eine separate Titelseite geben soll oder nicht.
															% leqno Die Nummer bei nummerierten Formeln soll links, statt rechts, dargestellt werden.
															% fleqn Formeln sollen linksb�ndig statt zentriert dargestellt werden.
															% openbib Es soll das �offene� Bibliographie-Format verwendet werden.
final,														% draft, final Legt fest, ob es sich bei dem Dokument um einen Entwurf oder um die ?nale Version handelt. Das wirkt sich auf verschiedene Pakete aus. Beim Entwurf werden z. B. Bilder nur als Rahmen dargestellt, und �bervolle Boxen werden mit einer Linie markiert.
oneside,													% oneside, twoside W�hlt, ob die Ausgabe auf doppelseitigem oder auf einseitigem Papier erfolgen soll.
openany,													% openright, openany De?niert, wo neue Kapitel beginnen d�rfen. Mit �openright� werden neue Kapitel nur auf einer rechten Seite begonnen.
onecolumn,													% onecolumn, twocolumn Legt fest, ob der Text ein- oder zweispaltig gesetzt werden soll.
%bibliography = totoc
]{scrreprt}       % Dokumentklasse ggfs. anpassen
%draft oder final
%[DIVcalc,BCORY ]
%scrbook, scrreprt, scrartcl

%
% Maketitle-Informationen
%
\titlehead{Simulationsmethoden in der Physik \hfill Computerphysik II}
\subject{Report}
\title{Molekular-Dynamik von weichen Teilchen}
\subtitle{}
%\author{Maximilian R�hl\footnote{maxi.ruehl@googlemail.com} $\quad$ y Sebastian Eibl\footnote{sebastian.eibl@sebastian-eibl.de}}
\author{Sebastian Eibl}
\date{\today}
\publishers{Friedrich Alexander Universit�t Erlangen-N�rnberg}
%\and
%\thanks{Fu�note }

%\dedication{Dies ist mein inoffizieller Reisebericht f�r meine Freunde! Nicht zur Weitergabe gedacht! Da ich im Moment noch relativ viel schreibe und auch versuche alles m�glichst zeitnah zu ver�ffentlichen, bitte ich �ber eventuelle Rechtschreib- und Grammatikfehler hinwegzusehen. F�r Korrekturen bin ich aber immer dankbar. Ich gebe mir viel M�he bei den spanischen Begriffen, da ich aber selber noch Spanisch lerne, kann ich nicht f�r die Richtigkeit garantieren. Ich hoffe, dass ich von meinen treuen Lesern, die besser Spanisch k�nnen als ich, darauf hingewiesen werde, wenn ich groben Unfug mache. ;-) Viel Spa� beim Lesen!}

%
% Sprachanpassungen
%
\usepackage[ansinew]{inputenc}        % ggfs. anpassen -- Eingabekodierung
\usepackage[T1]{fontenc}              % Zeichensatzkodierung
\usepackage[ngerman]{babel}           % Sprache einstellen
% \hyphenation{Ba-na-ne}     Trennung vorgeben

%
% Schriftart
%
\usepackage{lmodern}
%\usepackage{times}
%\usepackage{mathpazo}                % times font plus extra support for maths
%\usepackage{courier}
%\usepackage{helvet}                  % helvetica as the sans serif font
					 
%
% Erweiterungen
%
\usepackage{paralist}								%kompaktere Aufz�hlungen mit compactenum und compactitem
%\usepackage[numbers,round]{natbib}
%\bibpunct{[}{]}{;}{n}{}{,}
\usepackage[german]{fancyref}        % sch�nere Referenzen
%\Fref{Label} und \fref{label}
\usepackage{graphicx}                % Bilder einbinden
%\includegraphics[Optionen ]{path/file}
%[iwidth,iheight,fscale,fangle,bclip,viewport]
\usepackage{booktabs}                % sch�nere Tabellen
%\toprule \midrule \bottomrule
\usepackage{amsmath}         		 % Mathe-Erweiterungen
\usepackage{amsfonts}
\usepackage{amssymb}
%\usepackage{textcomp}        		 % Weitere Symbole
%\usepackage{xcolor}          	     % Farbe
\usepackage{ragged2e}                % b�ndige Texte mit Silbentrennung
% \RaggedLeft, \RaggedRight, \Centering
%\usepackage{tikz}
%\usepackage{multicol}                % Text nebeneinander
% \begin{multicols}{<spaltenzahl>}[text_davor]
\usepackage[%
%headtopline,
%plainheadtopline,
%headsepline,
%plainheadsepline,
%footsepline,
%plainfootsepline,
%footbotline,
%plainfootbotline
]{scrpage2}                % flexiblere Kopf und Fu�zeilen
\usepackage[load-configurations=abbreviations]{siunitx}                 % einheitliche Einheiten
%SI[]{Wert}[]{Einheit}
\DeclareSIUnit\Ch{Ch}
%\usepackage{float}                   % Floatingumgebung
\usepackage{hyperref}                % verbessert die PDF-Ausgabe
%\href{URL}{Text}
%\url{URL}
\usepackage{isotope}

%
% Bug-Fixes
%
\usepackage{fixltx2e}        % Bugs korrigieren
\usepackage{mparhack}        % Marginpars korrigieren
\usepackage[toc,page]{appendix}
\usepackage[section]{placeins}	%get figures right

%
% href-Optionen
%
% \hypersetup{%
% pdftitle={NumerischeSimulationderintergranularen
% Ri�ausbreitungdurchSpannungskorrosion},
% pdfsubject={DiplomarbeitamIBNM},
% pdfauthor={SaschaBeuermann},
% pdfkeywords={Simulation,intergranualareRi�ausbreitung,
% Spannungskorrosion},
% pdfcreator={Adobe-Acrobat-Distiller},
% pdfproducer={LaTeXwithhyperrefandthumbpdf}
% }
\hypersetup{%
colorlinks=true,
}

% Absatzeinr�ckung durch Leerzeile ersetzen
\setlength{\parindent}{0pt}
\setlength{\parskip}{\baselineskip}

%deutsche Anpassung f�r siuntisx
\sisetup{
list-final-separator={ und },
range-phrase={ bis },
separate-uncertainty = true,
per-mode = fraction,
output-decimal-marker = {,}
}

%
% Einstellungen f�r scrpage2
%
\pagestyle{scrheadings}
\clearscrheadings
\clearscrplain
\clearscrheadfoot

%\renewcommand{\chaptermark}[1]{\markboth{#1}{}}
%\renewcommand{\chaptermark}[1]{\markright{#1}{}}

\chead{}
\ohead{}
\ofoot{}
\cfoot{\pagemark}

%\addto\captionsngerman{\renewcommand{\figurename}{Abb.}}

\usepackage{microtype, textcomp}       % als letztes -- Verbesserter Textsatz